\documentclass[aspectratio=169, xcolor=dvipsnames]{beamer}

\definecolor{mygray}{gray}{0.95}
\setbeamercolor{background canvas}{bg=}%transparent canvas
%\usetheme{Madrid}
\usetheme{Boadilla}
\setbeamertemplate{navigation symbols}{}
%\usecolortheme{seahorse} %seahorse seagull
\usecolortheme[named=Turquoise]{structure}
%\usecolortheme[named=Mahogany]{structure}
\usepackage{graphicx}
\usepackage{tikz}

\definecolor{egyptianblue}{rgb}{0.06, 0.2, 0.65}
\usepackage{xcolor}

\setbeamercovered{transparent}
\usepackage{tikz}

\usepackage{natbib}

\usepackage{xcolor}



%\usepackage{animate}

%\setbeamercovered{invisible}

\begin{document}




\title[A Self-supervised Approach for Cloud Image Analysis]{\textcolor{darkgray}{\tiny{22$^\mathrm{nd}$ Conference on Artificial Intelligence for Environmental Science}}\\
\Large{\textbf{A Self-supervised Approach for Cloud Image Analysis}}}  
\author[NSF SAGE]{\textcolor{olive}{\scriptsize{Bhupendra A. Raut, Dario Dematties, Robert C. Jackson, Seongha Park, Sean Shahkarami, Yongho Kim,\\ Rajesh Sankaran, Pete Beckman, Scott Collis, and Nicola Ferrier}}}
\institute[Northwestern-Argonne Institute]{\textcolor{WildStrawberry}{\tiny{Northwestern-Argonne Institute of Science and Engineering, 
Northwestern University, Evanston, IL\\
Argonne National Laboratory, Lemont, IL}}}


\date[103rd AMS-Denver | 11 January 2023]{\textcolor{darkgray}{\scriptsize{11 January 2023}}}

%{
%\usebackgroundtemplate{\includegraphics[width=\paperwidth, height=\paperheight, page=1]{./bg1.jpg}}%
\frame[plain]{\includegraphics[height=1.5cm]{sage.jpg}
\hfill\includegraphics[height=1.5cm]{ams.png}\titlepage
\centering
\includegraphics[height=0.8cm]{northwestern_logo}\hfill
\includegraphics[height=1.2cm]{nsf.png}\hfill
\includegraphics[height=1cm]{Argonnelablogo.PNG}} 
%}

%1. add NAISE and Argonne affiliation.
%2. add AMS and AI conference names on the front page.
%3. add final slides with SAGE logo.



\section{Introduction}
\frame{\frametitle{Complexity of Cloud Structures at Different Scales}
\includegraphics[width=0.4\textwidth, height=0.49\textwidth]{fig/sicily_tmo_2022341_lrg.jpg}
\includegraphics[width=0.5\textwidth]{fig/fig-crop.pdf}
}

\frame{\frametitle{Practical difficulties in supervised learning}
\begin{itemize}
\item Supervised learning requires labeled datasets.
\item Labeling is expensive and tedious, especially for the segmentation of cloud images.
\item More advanced models are more data-hungry. Perceptrons $<$ CNNs $<$ Transformers 
\end{itemize}

\textbf{Solution:} Generally, pre-trained models (ImageNET) are fine-tuned with a smaller domain-specific dataset.

\begin{exampleblock}{Better Solution}
Models are self-trained without labels and later fine-tune with a small labeled dataset.
\end{exampleblock}

}


\frame{\frametitle{Joint embedding architecture for self-supervised learning}
\begin{columns}

\begin{column}{0.5\textwidth}
\includegraphics[width=\textwidth, height=\textheight, trim={0cm 0 60cm 0},clip]{fig/Model_Architecture.png}
\end{column}

\begin{column}{0.5\textwidth}
Distillation with no labels (DINO) \citep{caron2020unsupervised, caron2021emerging}\vskip 0.2in
How it is intended to use?
\begin{enumerate}
\item Train with raw data first, then fine-tune it with labeled data as needed.
\item Any NN can be embedded with this architecture. Transformers, CNN, Encoder-Decoders.
\item We can ask to ignore certain features. In this case `color order' and `values'.
\end{enumerate}
This is great news for training models at the Edge (SAGE project, \citet{beckman2016waggle}) and also for satellite meteorology.
\end{column}

\end{columns}
}


\frame{\frametitle{Self-supervised learning for cloud segmentation and classification}
\includegraphics[width=\textwidth, height=0.8\textheight]{fig/Model_Architecture.png}
}


\frame{\frametitle{Self-supervised learning for classification (PCA1 Vs PCA2)}
\begin{columns}
\begin{column}{0.6\textwidth}
\includegraphics[width=0.9\textwidth, height=0.8\textheight]{fig/Zones.png}
\end{column}

\begin{column}{0.4\textwidth}
\begin{itemize}
\item The clear sky and cloudy images have occupied two large regions.
\item The overcast sky images with low and mid-level clouds are also separated into two regions.
\item Partly cloudy images tend to have multiple regions depending on their structure.
\end{itemize}

\end{column}

\end{columns}
}





\frame{\frametitle{Self-supervised learning for classification (SOM)}

\begin{columns}

\begin{column}{0.5\textwidth}
\includegraphics[width=\textwidth, height=0.9\textheight]{fig/SOM_5_4_N.png}
\end{column}


\begin{column}{0.5\textwidth}
\begin{itemize}
\item The clusters are characterized by different cloud types, and cloud coverage and have different diurnal peaks. 
\item More results are presented in  \citet{dario2023cloudino}, `Let’s unleash the network judgement: A self-supervised approach for cloud image analysis.' Artificial Intelligence for the Earth Systems, In review.
\end{itemize}
\end{column}

\end{columns}
}





\frame{\frametitle{Self-supervised learning for classification (SOM)}

\begin{columns}

\begin{column}{0.5\textwidth}
\includegraphics[width=\textwidth, height=0.9\textheight]{fig/SOM_5_4_N.png}
\end{column}


\begin{column}{0.5\textwidth}
\vskip0.2in
\tiny{Clouds tend to form height-wise clusters \citep{dario2023cloudino}.}
\includegraphics[width=\textwidth, height=0.8\textheight]{fig/cbhHist_firstCBH_10min.pdf}\\
\end{column}

\end{columns}
}








\frame{\frametitle{Attentional maps for segmentation without labels}
\begin{columns}

\begin{column}{0.5\textwidth}
\includegraphics[width=\textwidth, height=0.6\textheight]{fig/Segmentation.png}

\end{column}

\begin{column}{0.5\textwidth}

\begin{itemize}
\item Relative values of attention are normalized to get the cloud/no cloud segmentation.
\item Attention values for clouds tend to match the transparency of the clouds.
\item Caution: Interpreting the attention values as transparency is complicated. 
\end{itemize}

\includegraphics[width=\textwidth, height=0.45\textheight]{fig/Segmentation_Perf_Stats_per_Cluster.png}
\end{column}

\end{columns}
}



\frame{\frametitle{Attentional maps differentiating clouds at different heights.}
\begin{columns}

\begin{column}{0.35\textwidth}
\includegraphics[width=\textwidth, height=\textwidth]{qr-code.pdf}\\
\href{https://youtube.com/shorts/LOmXbf--U20?feature=share}{Click for YouTube Short}:$<1$min animation, no sound.
\end{column}

\begin{column}{0.3\textwidth}
\begin{itemize}
\item Scan the QR-code to see an animation of different heads paying attention to the clouds at different levels.
\item During multilayer scenario, lower and upper level clouds are seen in left and right bottom heads, respectively. $\Longrightarrow$
\end{itemize}
\end{column}

\begin{column}{0.35\textwidth}
\includegraphics[width=\textwidth, height=1.2\textwidth]{fig/heads.jpeg}\\
\end{column}

\end{columns}
}








\section{Applications and Future Development}
\frame{\frametitle{Applications and Future Development}
\begin{columns}

\begin{column}{0.4\textwidth}
\includegraphics[width=0.8\textwidth, height=\textwidth]{fig/sat.png}
\end{column}

\begin{column}{0.6\textwidth}
The joint embedding architecture and Vision Transformers show promising results in cloud classification and characterization, with potential applications for large datasets of satellite images.


Future Development:\\
\begin{itemize}
\item Use of the model on satellite images, including multichannel images for more complex feature extraction.
\item To provide a pre-train model with millions of satellite images to classify cloudiness across the globe.
\item Flexibility in using different model architectures (e.g. Convolutional Neural Networks, Encoder-Decoder)
\end{itemize}
\end{column}

\end{columns}
}


\frame{\frametitle{References}
\tiny
\bibliographystyle{ametsoc2014}
\bibliography{/Users/bhupendra/myPapers/bibtex_library/Convection.bib}
}

{\usebackgroundtemplate{\includegraphics[width=\paperwidth, height=0.95\paperheight]{sage-sponsors.pdf}}%
\frame{\frametitle{}
\centering \includegraphics[height=1.5cm]{ams.png}
}
}











\end{document}
